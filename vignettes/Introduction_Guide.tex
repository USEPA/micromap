\documentclass{article}

\usepackage{hyperref}

% knitr and global options

\newcommand{\R}{{\normalfont\textsf{R }}{}}
\newcommand{\NA}{{\normalfont\textsf{NA }}{}}

%\VignetteIndexEntry{Introduction to the micromap package}

\usepackage{Sweave}
\begin{document}

\input{Introduction_Guide-concordance}

\title{Linked Micromaps}

\author{Quinn Payton, Marc Weber, Michael McManus, Tony Olsen, Tom Kincaid}
\maketitle


\section{Introduction}
The \R package \verb@micromap@ is used to create linked micromaps, which display statistical summaries associated with areal units, or polygons. Linked micromaps provide a means to simultaneously summarize and display both statistical and geographic distributions by linking statistical summaries to a series of small maps. The package contains functions, heavily dependent on the utilities of the \verb@ggplot2@ package, which may be used to produce a row-oriented graph composed of different panels, or columns, of information. These panels at a minimum contain maps, a legend, and statistical summaries.

The key to using these functions is to have your data set up correctly. For a first example, we would like to display US state names, a graph illustrating their poverty level, a graph illustrating their percentage of college graduates, and a micromap indicating which states are being referenced. In order to do this, all we need is a table with state names and estimates of each of the two metrics we're interested in. The dataset \verb@edpov@ included in the \verb@micromap@ library is in this form:

\begin{Schunk}
\begin{Sinput}
> library(micromap)